\documentclass[11pt,a4paper,sans]{moderncv}


\moderncvstyle{classic}
\moderncvcolor{blue}

\usepackage[top=2cm,bottom=2cm,left=2cm,right=2cm]{geometry}
\usepackage{xspace}

\name{Paul}{Li\'etar}
%\address{26 chemin saint Jean}{38700 La Tronche}{France}
%\phone[mobile]{+33~7~78~21~53~78}
%\phone[fixed]{+33~4~76~49~91~83}
\phone[mobile]{+44~7518~295042}
\email{paul@lietar.net}
%\social[github]{plietar}
%\social[linkedin]{plietar}
%\extrainfo{Born on December 28th, 1996}

\setlength{\hintscolumnwidth}{0.18\textwidth}
\newcommand{\Cpp}{\texttt{C++}\xspace}

\begin{document}
\makecvtitle

\section{Experience professionelle}
\cventry{2022-2023}{Ing\'enieur de recherche}{Microsoft Research}{}{Cambridge, UK}{}
\cventry{2021-2022}{Ing\'enieur logiciel}{Cloudflare}{}{Londres}{
    I was an engineer on Quicksilver, the distributed key-value store that
    powers Cloudflare's edge network, replicating customer configurations from
    the core datacenter to all servers around the world.
    I worked on a project to partition the database across multiple nodes and
    played a large role in in designing a scheme to provide a consistent view
    to applications, even in the face of asynchronous replication. I lead
    the development of a new caching layer to optimize median request latency.
}
\cventry{2019\\9 months}{Chercheur Stagiaire}{Microsoft Research}{}{Cambridge, UK}{
    I participated in the design of Verona, a programming language for safe
    infrastructure programming. I designed and implemented the original
    compiler and bytecode interpreter, which has been used to show-case the
    language both internally and externally.\\
    I optimized and benchmarked \textit{snmalloc}, the memory allocator on top
    of which the Verona runtime is built. I was the first author on our paper
    describing the allocator's operation, and presented the results at the ISMM
    conference in Phoenix, AZ.
}
\cventry{2017-2018}{Ing\'enieur logiciel}{Google}{Counter-Abuse Technology}{Zurich}{
    I worked on Android’s device integrity solution, protecting Google’s
    services against automated fraud and abuse. I worked both on the
    server side infrastructure, serving billions of devices around the world,
    and on the client side powering said devices.
}
\cventry{\'Et\'e 2016\\3 mois}{Stagiaire en g\'enie logiciel}{Google}{Android Security}{Mountain View}{
    I investigated the use of Rust to write Android services, with the
    long-term vision of replacing the use of \Cpp in low-level, performance
    sensitve components. I focused on the interopability between newly written
    Rust code and existing \Cpp and Java code, both intra- and inter-process. %\\
    % For the latter, I implemented a Rust library to use the binder IPC
    % mechanism present in the Android devices, by communicating with the
    % corresponding kernel driver through ioctl. This allowed new services to be
    % written entirely in Rust, while still being able to interface with services
    % and applications written in Java and \Cpp.
}

\cventry{\'Et\'e 2015\\3 mois}{Stagiaire en g\'enie logiciel}{Google}{Certificate Transparency}{London}{
    I added support for Certificate Transparency to Android's network stack.
    This required implementing the OCSP and SCT extensions to BoringSSL and
    Conscrypt, the TLS implementations used by Android applications. I added
    an API allowing Android applications to opt-in to Certificate Transparency
    verification when establishing new network connections.
}

\cventry[1em]{Hiver 2013/14\\5 mois}{Software Developer}{DecaWave}{Dublin}{}{
    Contributed to the development of DecaWave's Real Time Location System
    software. I worked both on the firmware running on ARM Cortex M3 devices,
    interfacing with DecaWave's custom silicon, as well as on the development
    of the coordination software on desktop, productionizing the
    algorithms designed by the research team.
}

%\cventry[1em]{July 2013, 4 weeks\\July 2012, 3 weeks}{Software Developer}{Movea}{Grenoble}{}{
%    Development of applications for Android devices, demonstrating the use cases of products made by Movea,
%    a leading motion sensor data fusion start-up, recently acquired by Invensense.
%}

%\cventry[1em]{Spring 2013}{Volunteer}{Hexagone}{Grenoble}{}{
%    Development of an iOS application for l'Atelier Arts Sciences.
%}
%
%\cventry[1em]{July 2011\\3 weeks}{Intern}{Parrot}{Paris}{}{
%    Rewrite of an iPhone game from Unity to a lightweight OpenGL engine for Parrot's drones.
%}

%\cventry{December 2009\\1 week}{Intern}{ST Microelectronics}{Agrate}{Italy}{
%    Development of a Symbian S60 application to display images from a network camera.
%}

\section{Publications}
\cventry{Juin 2019}{Snmalloc: a message passing allocator}{}{}{}{
    \textit{Proceedings of the 2019 ACM SIGPLAN International Symposium on Memory Management}
}

\section{Education}
\cventry{2019-2021}{PhD Program (unfinished)}{Imperial College}{London}{}{
    I worked on the design and formalization of a type system for regions,
    bringing safe and performant memory management to systems programming.
    The type system forms the foundations of the Verona programming language.
}

\cventry{2014-2017}{BEng Mathematics and Computer Science}{Imperial College}{London}{}{
    Obtenu avec la plus haute distinction (\textit{first class honours}).
    Mon m\'emoire de recherche, \textit{Formalizing Generics in Pony}, fut distingu\'e et m'a valu de recevoir le prix \textit{Corporate Partnership Programme Award}, decerne a un eleve de derniere annee pour 

    My bachelor thesis on \textit{Formalizing Generics in Pony} was selected as a distinguished project, and I
    received the , awarded to final year students for outstanding
    achievement in their individual projects.
}

% \cventry{2010-2013}{French Baccalaureate}{}{}{}{With Honors in Mathematics}


%\section{Projects}
%\cventry[1em]{Spring 2016}{Outgain}{Imperial College 2nd Year Group Project}{}{}{
%    Online multiplayer game where users write an artificial intelligence, with an educative focus.\\
%    \textit{IBM 2nd Year Group Project Prize} for best third term project in a second year project group.\\
%    \url{https://github.com/egnwd/outgain}
%}
%
%\cventry[1em]{}{librespot}{Personal Project}{}{}{
%    Reverse engineered, open source, Spotify client.\\
%    \url{https://github.com/plietar/librespot}
%}
%
%\cventry[1em]{}{Contributions to Open Source Projects}{}{}{}{}

%\pagebreak
%
%\section{Activities, achievements and personal experience}
%\cventry[1em]{2009 to present}{Competitive Rowing}{}{}{}{
%    Bronze medal at French national championship in 2010 in boat of eight.\\
%    Raced for Imperial College at the British University Championships in 2015 \& 2016.
%}
%
%\cventry[1em]{Spring 2014\\Spring 2016}{Backpack travels on my own}{}{}{}{
%    \begin{itemize}
%        \item Visiting Italy for three weeks
%        \item Cycling across Corsica for a week
%        \item Cross country skiing in northern Sweden for a week
%    \end{itemize}
%}
%
%\cventry[1em]{}{Finalist at Prologin, the under 21 French national computing contest}{}{}{}{12th in 2011 and 67th in 2012.}
%
%\cventry[1em]{}{Lived abroad with family in Singapore (2 years) and California (4 years)}{}{}{}{}

%\section{Technical skills}
%\cvitem{Languages}{C, Python, Java, Haskell, Go, Rust}
%\cvitem{Platforms}{PC, Android, STM32}

%\section{Languages}
%\cvitem{French}{Native language}
%\cvitem{English}{Fluent, iBT Toefl 108/120}
%\cvitem{Spanish}{Basic}

%\addvspace{1em}
%\cventry{}{\textit{References upon request}}{}{}{}{}

%\makecvfooter
\end{document}
